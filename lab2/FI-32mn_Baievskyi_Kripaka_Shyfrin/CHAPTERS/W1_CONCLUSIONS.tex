%!TEX root = ../thesis.tex
% створюємо Висновки до всієї роботи
Підключені пристрої та їхні комунікації мають бути захищені від загроз і атак, що розвиваються, для захисту екосистем і цінної особистої та ділової інформації. Високоякісні TRNG є фундаментальною технологією, необхідною для побудови ланцюжка довіри в системах, оскільки багато криптографічних алгоритмів шифрування/автентифікації та протоколів безпеки залежать від істинних випадкових чисел для генерації ключів, викликів і початкових значень. Загальна безпека систем і додатків залежить від якості джерела ентропії, яке забезпечують TRNG. Недоліки генераторів випадкових чисел можуть бути використані зловмисниками для компрометації пристроїв, які в іншому алгоритмічно безпечні. Ефективні ГВЧ повинні відповідати стандартам NIST і AIS і можуть бути сертифіковані в кінцевих продуктах за такими сертифікатами, як FIPS 140-2/140-3, Common Criteria (CC) і китайський OSCCA.


Генерація випадкових чисел це необхідне і складне завдання, для якого пропонується безліч рішень. Однак будь-яка сучасна обчислювальна система (включно з IoT-пристроями), якщо їй потрібні випадкові числа, матиме у своєму складі TRNG. Хоча мінімальна ентропія, яку можна витягти з кожного типу випадкових подій за певний проміжок часу, може сильно відрізнятися для кожного з них. TRNG самі по собі зазвичай не підходять для багатьох криптографічних цілей, оскільки істинність їхньої випадковості не має на увазі незміщену рівномірність їхніх виходів. Ця нерівномірність може бути особливо важливою під час переведення зі сфери вимірювань у сферу необхідної випадковості.


Однак, сучасні вимоги безпеки стикаються з обмеженнями генераторів істинних випадкових чисел (\cite{cryptoSEQuestion}, \cite{goubin2000true}, \cite{chmielewski2007true}). Використання фізичних джерел у TRNG призводить до високого енергоспоживання й обмеженої пропускної здатності, що стає неприйнятним для сучасних інтегрованих систем. TRNG також чутливі до змін умов експлуатації, що вимагає додаткової постобробки для забезпечення стабільності вихідних даних.


Але підхід майже завжди полягає у використанні як TRGN для вибірки випадковості, так і PRGN або DRGN для змішування нової випадковості зі збереженою випадковістю в унікальні, рівномірні та статистично незалежні вихідні дані, які не піддаються вгадуванню. У такій системі, яка добре спроектована і реалізована, доступ до будь-якої кількості виходів не повинен допомагати в вгадуванні минулих або майбутніх виходів. Якщо в системі використовується тільки TRGN без будь-якого PRGN, то варто ставитися з великою підозрою до такої системи. Загальноприйнято, що чим швидше, рівномірніше розподілені, ентропійніше й ефективніше, тим краще. Але сенс спільного використання TRGN і PRGN, можливо, для створення CSPRNG, полягає в наступному: Після отримання достатньої кількості вихідної випадковості швидкість, рівномірність і ентропію можна досягти шляхом постійного поповнення пулів навіть досить низькоякісними TRGN, а потім вилучення вихідних даних за допомогою ефективного PRGN, наприклад, легкого симетричного шифру або хеш-функції. Так, наприклад, алгоритм Hash DRBG з використанням SHA-256 може бути обраний для розробки ядра CSPRNG. Цей вибір забезпечує необхідний рівень безпеки за прийнятної пропускної здатності.
