%!TEX root = ../thesis.tex
% створюємо вступ
\section{Мета практикуму}

Дослідити можливість реалізації розділення секрету за допомогою різних асиметричних алгоритмів (не менше як двох) та порівняти їх ефективність за обраним критерієм.

\subsection{Постановка задачі та варіант}
\begin{tabularx}{\textwidth}{X|X}
	\textbf{Треба виконати} & \textbf{Зроблено} \\
	Дослідити існуючі схеми розділення секрету & \checkmark \\
	Дослідити схеми розділення в різних асиметричних алгоритмах & \checkmark \\
        Порівняти їх ефективність за обраним критерієм & \checkmark \\
\end{tabularx}

\section{Хід роботи/Опис труднощів}
    На початку роботи над практикумами вибрати варіант 1А, та далі продовжували роботу над ними. Згідно вибраного варіанту у даній роботі буде розглянуто схеми розділення секрету в різних асиметричних алгоритмах. Під час виконання звіту не виникло ніяких серйозних проблем.